%%%%%%%%%%%%%%%%%%%%%%%%%%%%%%%%%%%%%%%%%%%%%%%%%%%%%%%%%%%%%%%%%%%%%%
%%  Copyright by Wenliang Du.                                       %%
%%  This work is licensed under the Creative Commons                %%
%%  Attribution-NonCommercial-ShareAlike 4.0 International License. %%
%%  To view a copy of this license, visit                           %%
%%  http://creativecommons.org/licenses/by-nc-sa/4.0/.              %%
%%%%%%%%%%%%%%%%%%%%%%%%%%%%%%%%%%%%%%%%%%%%%%%%%%%%%%%%%%%%%%%%%%%%%%

\input{../../common-files/header}
\input{../../common-files/copyright}

\newcommand{\dnsFigs}{./Figs}
\lhead{\bfseries SEED Labs -- 远程DNS缓存中毒攻击实验}


\def \code#1 {\fbox{\scriptsize{\texttt{#1}}}}

\begin{document}
\begin{center}
{\LARGE 远程DNS攻击 (Kaminsky 攻击) 实验}

\vspace{0.05in}
Updated on July 26, 2020
\end{center}

\seedlabcopyright{2006 - 2020}


% *******************************************
% SECTION
% ******************************************* 
\section{实验综述}




本实验的目的是让学生掌握远程DNS缓存中毒攻击的经验,也被称为Kaminsky DNS攻击。
DNS(Domain Name System)是互联网的电话本,它将域名转换为IP地址,反之也亦然。
这种转换是通过DNS解析完成,这个解析过程往往用户并不知道。DNS攻击便是操纵这一解析过程,
以各种方式将用户误导至设计好的目的IP,这个目的IP往往是恶意的。
本实验关注于一种特定的DNS攻击技术,称为{\em DNS 缓存中毒攻击}。
在另一个SEED Lab中,我们设计了一系列活动来进行本地网络环境下的DNS缓存中毒攻击,
即攻击者和受害DNS服务器位于同一网络中,在这种情况下,可以使用数据包嗅探的技术。
而在远程攻击的实验中,数据包嗅探技术不再可用,因此远程DNS攻击变得更有挑战和难度。
本实验涵盖以下主题:


\begin{itemize}[noitemsep]
\item DNS介绍与DNS工作原理
\item DNS服务器搭建
\item DNS缓存中毒攻击
\item 伪造DNS响应
\item 数据包伪造
\end{itemize}


\vspace{0.1in}
\newnote{本实验在2020年7月26日进行了修订,实验配置已被大量修改,
并作为实验任务的一部分。因此,实验任务内容已更改。
}


\paragraph{相关阅读材料与视频:}
关于DNS协议与攻击的详细内容可以参考以下材料:

\begin{itemize}
\item Chapter 18 of the SEED Book, \seedbook
\item Section 7 of the SEED Lecture, \seedisvideo
\end{itemize}


\paragraph{实验环境} \seedenvironment


\vspace{0.2in}
\noindent
\fbox{\parbox{\textwidth}{
\noindent
\textbf{用户定制:}
在此次实验描述中,我们使用\texttt{attacker32.com}域作为攻击者控制的恶意域名。
当学生在进行本实验时,不被允许使用这个域名,学生应当使用一个包含自己名字的域名。
该要求的目的是区分学生的工作,由于该域名只在实验环境中可见,并不公开使用,因此任何域名
(即使已被他人拥有的域名)都可以在实验环境中安全地使用。
}}




% *******************************************
% SECTION
% ******************************************* 
\section{实验环境搭建任务}
\label{sec:environment}

\begin{figure}[htb]
\centering
\includegraphics[width=0.9\textwidth]{\dnsFigs/environment_setup_remote.pdf}
\caption{实验环境搭建}
\label{dns:fig:environment}
\end{figure}


DNS缓存中毒攻击的主要目标是本地DNS服务器。显然,攻击真实的DNS服务器是违法的,因此
我们需要搭建自己的DNS服务器来进行攻击实验。本次实验环境需要三台主机:受害者主机,DNS服务器和
攻击者主机。我们在单个宿主机上运行这三个虚拟机,这些VMs都使用我们预先构建的\ubuntu VM镜像。
图~\ref{dns:fig:environment}说明了实验环境的设置。如果使用{\tt VirtualBox},请使用
{\tt "NAT Network"}作为每个VM的网络适配器选项,如果使用{\tt Vmware},使用默认的{\tt "NAT"}即可。
为了简单起见,我们将这些VMs都放在同一LAN中,学生不能在攻击中利用这个便利,他们应该将攻击者的机器
视为远程机器,即攻击者无法在LAN上嗅探数据包。


在以下各节中,我们假定用户计算机的IP地址为{\tt 10.0.2.6},本地DNS服务器的IP地址为{\tt 10.0.2.7},
攻击者主机的IP地址为{\tt 10.0.2.8},我们称本地DNS服务器为\texttt{Apollo}。



% -------------------------------------------
% SUBSECTION
% ------------------------------------------- 
\subsection{任务 1: 配置用户主机} 
\label{subsec:user_machine}


在用户主机{\tt 10.0.2.6}上,我们需要使用{\tt 10.0.2.7}作为本地的DNS服务器。
这是通过更改用户主机的解析配置文件~(\texttt{/etc/resolv.conf})来实现的,将
\texttt{10.0.2.7}添加为第一个\texttt{nameserver}条目,即该服务器将用作主DNS服务器。
然而,由于我们提供的VM都采用Dynamic Host Configuration Protocol (DHCP)来获取网络配置参数,例如IP地址,本地DNS服务器等,
DHCP 客户端会用从DHCP服务器获取的网络配置信息覆盖并重写\texttt{/etc/resolv.conf} 文件。


为防止我们在 \texttt{/etc/resolv.conf} 文件中的配置被DHCP覆盖,我们可以添加以下内容到
\path{/etc/resolvconf/resolv.conf.d/head}文件中(假设\texttt{10.0.2.7}是本地DNS服务器的IP地址):


\begin{lstlisting}
nameserver 10.0.2.7
\end{lstlisting}


Head文件的内容会加在自动生成的配置文件之前,通常来说只是一个注释行 ( \texttt{/etc/resolv.conf} 的注释往往来自其head文件 ) 。
在完成修改之后,我们需要执行以下命令来使修改生效:

\begin{lstlisting}
$ sudo resolvconf -u
\end{lstlisting}



\paragraph{测试:}

在成功配置完用户主机后,使用\texttt{dig}命令来尝试获取一个主机名(hostname)的IP地址,从dig的回显来查看
响应内容是否是从本地的DNS服务器返回,如果不是则证明未成功配置。



% -------------------------------------------
% SUBSECTION
% ------------------------------------------- 
\subsection{任务 2: 配置本地DNS服务器 (the Server VM)} 


针对本地DNS服务器,我们需要运行一个DNS服务软件,目前用的最广的DNS服务软件是BIND~(Berkeley Internet Name
Domain),该软件最初由加州大学伯克利分校于1980年代初期开发设计,最新的版本
名为BIND 9,它于2000年首次发布。我们将在实验环境下展示如何配置BIND 9软件。
BIND 9软件已经预装在\ubuntu VM镜像中并会在系统开启时自启动。


BIND 9 从配置文件\path{/etc/bind/named.conf}读入软件配置,该文件是主配置文件,
它通常包含许多\texttt{"include"}入口,真正的配置内容往往保存在这些引入文件中。
其中我们通常需要设置的是\path{/etc/bind/named.conf.options}文件:


\paragraph{Step 1:删除 {\tt example.com} 区域:}
如果你做了 ``本地DNS攻击实验'',你很有可能已经配置了本地DNS服务器{\tt Apollo}来管理
{\tt example.com}域。在本实验中,该DNS服务器不再管理这个域,因此需要从{\tt /etc/bind/named.conf}
文件中删除相关的域设置。



\paragraph{Step 2: 配置一个正向区域:}
在本实验中,Kaminsky攻击的主要目的是使受害者使用 \texttt{ns.attacker32.com} 作为
\texttt{example.com}域的名称服务器。一旦攻击成功,受害DNS服务器会将对\texttt{example.com}域
的所有查询发送给\texttt{ns.attacker32.com}。 



在现实世界中,当本地DNS服务器需要查找\texttt{ns.attacker32.com}的IP地址。
它将前往根服务器 \texttt{com}服务器询问,并最终从管理\texttt{attacker32.com}域的名称
服务器获得响应。一旦本地DNS服务器得到这个IP地址,它会向这个IP地址发查询请求。这也是学生
会遇到问题的地方,因为学生不拥有\texttt{attacker32.com}域名(该域名事实上由本实验的作者杜文亮教授拥有),
因此学生将无法配置运行在\texttt{ns.attacker32.com}域的DNS服务器。


为解决这一问题,学生可以购买自己的域名,并在攻击中使用自己购买的域名,而不是\texttt{attacker32.com}。
这样,学生可以自己配置他们的DNS服务器来进行回答响应。然而这种方法对于学生来说成本太高了。
 

幸运的是,BIND9允许我们在DNS配置中添加一个正向区域。将以下区域条目添加到\path{/etc/bind/named.conf}文件中。
这项条目表示对于\texttt{attacker32.com}域的所有查询,都将转发到\texttt{10.0.2.8}。
这等效于将\texttt{10.0.2.8}作为\texttt{attacker32.com}域的名称服务器。
因此,使用这个配置,本地DNS服务器不会去查询\texttt{attacker32.com}域的IP地址,因为它已经有IP地址了。
请不要忘记配置中的所有分号,不然配置会失效。


\begin{lstlisting}
zone "attacker32.com" {
    type forward;
    forwarders {
        10.0.2.8;
    };
};
\end{lstlisting}
 


\paragraph{Step 3: 配置一些选项:} 
在我们的SEED虚拟机中,我们已经完成了所有的配置,这里只是希望学生可以了解到我们所做的配置内容。
如果学生使用的是SEED VM,那么这里不需要做任何操作。我们完成的配置内容在 \path{/etc/bind/named.conf.options}文件中。


\begin{itemize} 
\item 
\textbf{配置DNS缓存的转储位置:} 
以下配置指定了当BIND需要转储DNS缓存时,缓存文件的存储位置。如果该选项没有指定,
BIND将转储缓存到默认文件\path{/var/cache/bind/named_dump.db}。

\begin{lstlisting}
  options {
      dump-file "/var/cache/bind/dump.db";
  };
\end{lstlisting}

下面两个命令与DNS缓存相关,第一个命令将DNS缓存内容转储到之前指定的位置,第二个命令将清除缓存。
\begin{lstlisting}
$ sudo rndc dumpdb -cache    // Dump the DNS cache to the specified location
$ sudo rndc flush            // Clear DNS cache
\end{lstlisting}



\item 
\textbf{关闭 DNSSEC:} 
DNSSEC的引入为了防止DNS服务器受到欺骗攻击,
为了展示DNS攻击的效果,我们需要关闭这种保护机制。可以通过修改\path{named.conf.options}文件:
注释掉{\tt dnssec-validation} 行,添加一行
{\tt dnssec-enable} 。

\begin{lstlisting}
  options {
      # dnssec-validation auto;
      dnssec-enable no;
  };
\end{lstlisting}


\item 
\textbf{固定源端口号:}
DNS服务器现在在DNS查询中随机化源端口号,这使得攻击更加的困难。不幸的是,许多DNS服务器
仍然使用可以预测的源端口号。在本实验中,出于简单的考虑,我们假设源端口号是一个固定的数字。
我们可以将所有DNS请求的源端口设置为{\tt 33333}。可以通过在{\tt /etc/bind/named.conf.options}文件
中添加以下配置来生效:


\begin{lstlisting}
   query-source port 33333
\end{lstlisting}

\end{itemize}



\paragraph{Step 4: 重启DNS服务器:}
我们可以用以下命令重启DNS服务器。每次修改DNS配置后,DNS服务器都需要重启来使配置生效。
以下命令可以启动或重启\texttt{BIND 9}DNS服务器。

\begin{lstlisting}
$ sudo service bind9 restart
\end{lstlisting}




% -------------------------------------------
% SUBSECTION
% ------------------------------------------- 
\subsection{任务 3: 配置攻击者主机}

在攻击者主机,我们将管理两个区域。一个是攻击者合法的区域\texttt{attacker32.com},
另一个是假的\texttt{example.com}区域。


\begin{itemize} 
\item Step 1: 从实验网站下载\texttt{attacker32.com.zone}和
              \texttt{example.com.zone}文件。 

\item Step 2: 根据学生实际的网络配置修改这些文件(e.g., 一些IP地址需要修改)。 

\item Step 3: 将这两个文件拷贝到\texttt{/etc/bind} 目录。 

\item Step 4: 在\path{/etc/bind/named.conf}添加以下条目:


\begin{lstlisting}
zone "attacker32.com" {
        type master;
        file "/etc/bind/attacker32.com.zone";
};

zone "example.com" {
        type master;
        file "/etc/bind/example.com.zone";
};
\end{lstlisting}


\item Step 5: 重启DNS服务器。
\end{itemize} 
 


% -------------------------------------------
% SUBSECTION
% ------------------------------------------- 
\subsection{任务 4: 测试配置}

对于用户主机,我们运行一系列命令来确保我们的配置正确。


\paragraph{获取\texttt{ns.attacker32.com}的IP地址:}
当我们运行以下的\texttt{dig}命令时,由于在DNS配置中添加的\texttt{forward}区域,本地
DNS服务器会将查询请求转发到攻击者主机。因此,响应会从攻击者主机上配置的
\texttt{attacker32.com.zone}文件中返回。如果返回的结果不符合预期,那么配置过程可能存在问题。
请在实验报告中描述你的观察。


\begin{lstlisting}
$ dig ns.attacker32.com
\end{lstlisting}



\paragraph{获取\texttt{www.example.com}的IP地址:} 
目前有两个名称服务器管理\texttt{example.com}域,一个是这个域的官方名称服务器,另一个
是攻击者主机。我们将请求这两个名称服务器来查看会得到怎样的响应。
请运行以下两条命令(在用户主机上),并描述你观察到的现象。 


\begin{lstlisting}
// Send a request to the local DNS server, this request will go to the official name server of example.com
$ dig www.example.com

// Send a request directly to ns.attacker32.com
$ dig @ns.attacker32.com www.example.com
\end{lstlisting}
 


显然,没有人会向 \texttt{ns.attacker32.com}请求 \texttt{www.example.com}的IP地址,
人们会向 \texttt{example.com}域的官方名称服务器请求答案。所以DNS缓存攻击的目的是让
受害者去向\texttt{ns.attacker32.com}询问\texttt{www.example.com}的IP地址。顾名思义,
如果我们的攻击成功,当我们运行第一个\texttt{dig}命令时,即不带有 \texttt{@}选项的,我们应该
从攻击者主机得到一个虚假的结果,而不是从合法的名称服务器得到一个真实的结果。



% *******************************************
% SECTION
% ******************************************* 
\section{攻击任务}


DNS攻击的主要目的是在用户尝试使用$A$的域名前往$A$主机时,将用户重定向到另一个主机$B$。
例如,假设{\tt www.example.com}是一个在线银行网站,当用户尝试使用正确的URL {\tt www.example.com}
访问该网站时,如果攻击者可以将用户重定向到一个非常类似于{\tt www.example.com}的恶意站点,
那么用户很有可能被欺骗并向攻击者泄露自己的用户名密码。


在此任务中,我们将域名{\tt www.example.com}作为我们的攻击目标。值得注意的是, {\tt example.com}
域名保留供实验使用,不用作任何真实用途。 {\tt www.example.com}的真实IP地址为{\tt 93.184.216.34},
它的名称服务器由Internet Corporation for Assigned Names and Numbers (ICANN)管理。当用户
针对该域名运行{\tt dig}命令或在浏览器中输入该域名,用户主机会向本地DNS服务器发送DNS查询请求,
该DNS服务器最终将从{\tt example.com}的名称服务器中请求IP地址。


攻击的目标识对本地DNS服务器进行DNS缓存投毒攻击,例如当用户运行{\tt dig}命令来查找
{\tt www.example.com}的IP地址时,本地DNS服务器最终会进入攻击者的名称服务器{\tt ns.attacker32.com}
以获得IP地址,因此返回的IP地址可以是攻击者设计的任何值。结果,用户会被导向攻击者的恶意站点,而不是
真实的{\tt www.example.com}。




\begin{figure}[htb]
\centering
\includegraphics[width=0.9\textwidth]{\dnsFigs/DNS_Remote_new1.pdf}
\caption{完整的DNS查询过程} 
\label{fig:flow_diagram1}
\end{figure}


\begin{figure}[htb]
\centering
\includegraphics[width=0.9\textwidth]{\dnsFigs/DNS_Remote_new2.pdf}
\caption{Kaminsky攻击}
\label{fig:flow_diagram2}
\end{figure}



% -------------------------------------------
% SUBSECTION
% ------------------------------------------- 
\subsection{Kaminsky攻击原理}

在此任务中,攻击者向受害DNS服务器 ({\tt Apollo})发送DNS查询请求,从而触发来自
{\tt Apollo}的DNS查询。DNS查询首先前往其中一个根DNS服务器,接着 {\tt .COM}的DNS服务器,最终
从{\tt example.com}的DNS服务器得到查询结果,查询过程如图~\ref{fig:flow_diagram1}所示。
如果{\tt example.com}的名称服务器信息已经被{\tt Apollo}缓存,
那么查询不会前往根服务器以及{\tt .COM}DNS服务器,这个过程如图~\ref{fig:flow_diagram2}所示。
在本实验中,图~\ref{fig:flow_diagram2}描绘的场景更为常见,因此我们以这个图为基础来描述攻击机制。


当{\tt Apollo}等待来自{\tt example.com}名称服务器的DNS答复时,攻击者可以发送伪造的答复给{\tt Apollo},
假装这个答复是来自 {\tt example.com}的名称服务器。如果伪造的答复先到达,那么它将被
{\tt Apollo}接收,攻击成功。


如果你已经完成了本地DNS攻击的实验,你应该会意识到这些攻击都是假定攻击者和DNS服务器位于同一网段的,
即攻击者可以观察到DNS查询消息。当攻击者与DNS服务器不在同一网段,缓存投毒攻击变得非常困难。
主要的难点在于DNS响应中的传输ID(Transcation ID)必须与查询请求中的相匹配。由于查询中的传输ID
通常是随机生成的,在看不到请求包的情况下,攻击者很难猜到正确的传输ID。


显然,攻击者可以猜测传输ID。由于传输ID只有16位比特大小,如果攻击者可以在攻击窗口内伪造
$K$个响应(即在合法响应到达之前),那么攻击成功的可能性就是$K$/$2^{16}$。发送数百个伪造响应
并不是不切实际的,因此攻击者在成功之前不会进行太多的尝试。


然而,上述假设的攻击忽略了DNS缓存。在现实中,如果攻击者不幸没有在合法的响应到达之前猜中正确的传输ID,
那么DNS服务器会将正确的信息缓存一段时间。这种缓存效果使攻击者无法继续伪造针对该域名的响应,
因为DNS服务器在缓存超时之前不会针对该域名发出另一个DNS查询请求。为了继续对同一个域名的
响应伪造,攻击者必须等待针对该域名的另一个DNS查询请求,这意味着他必须要等到缓存超时。这个
等待时间可以是几小时或是几天。


\paragraph{Kaminsky攻击:} 
Dan Kaminsky提出了一个巧妙的方法来解决缓存的问题~\cite{dns:Kaminsky}。
通过他的方案,攻击者可以持续地发起欺骗攻击,而不需要等待,因此攻击可以在很短的
一段时间内成功。攻击的详细描述在~\cite{dns:Kaminsky,seedbook}可以找到。
在本任务中,我们将尝试这个攻击手段。以下步骤参考图~\ref{fig:flow_diagram2}概述了攻击的过程。


\begin{enumerate}
\item 攻击者向DNS服务器{\tt Apollo} 请求{\tt example.com}域中不存在的子域名,
如{\tt twysw.example.com},其中{\tt twysw}是一个随机的名字。 

\item 由于在{\tt Apollo}的DNS缓存中不可能匹配到这个子域名,因此{\tt Apollo} 
向{\tt example.com}域的名称服务器发送一个DNS查询请求。

\item 当{\tt Apollo} 等待答复时,攻击者向 {\tt Apollo}发送大量的伪造的DNS响应,
每个响应尝试一个不同的传输ID,希望其中一个是正确的。在响应中,攻击者不仅提供{\tt twysw.example.com}的
IP地址解析,攻击者还提供了一条``权威授权服务器''记录,其中指明{\tt ns.attacker32.com}是
{\tt example.com}域的名称服务器。如果伪造的响应比实际响应到达的早,且传输ID与请求中传输ID相匹配,
{\tt Apollo}就会接受并缓存伪造的答案,进而{\tt Apollo}的DNS缓存被投毒了。

\item 即使伪造的DNS响应失败了(例如,传输ID不匹配或到达的太晚了),
也没有关系,因为下一次攻击者会请求另一个子域名,所以{\tt Apollo}会发送另一个DNS请求,
从而给攻击者提供了另一个机会进行欺骗攻击。这种方法有效地克服了DNS缓存效果。



\item 如果攻击成功,那么在{\tt Apollo}的DNS缓存中,
{\tt example.com}域的名称服务器会被攻击者替换成{\tt ns.attacker32.com}。
为显示攻击的成功,学生需要展示在{\tt Apollo}的DNS缓存中有这样一条记录。
 


\end{enumerate}


\paragraph{任务综述:} 实现Kaminsky攻击十分具有挑战,因此我们将它分解为好几个子任务。在
任务4中,我们构造一个针对\texttt{example.com}域的随机子域名的DNS查询请求;在任务5中,我们构造一个从 \texttt{example.com}
名称服务器返回的伪造响应;在任务6中,我们把这些合在一起进行Kaminsky攻击;最后在任务7中我们验证攻击的效果。


% -------------------------------------------
% SUBSECTION
% ------------------------------------------- 
\subsection{任务 4: 构造DNS请求} 

这任务着重发送DNS请求。为了完成攻击,攻击者需要触发目标DNS服务器发出DNS查询,这样攻击者
有机会去欺骗DNS响应。由于攻击者需要尝试多次才能成功,因此最好使用程序来自动化该过程。

学生需要编写一个程序来向目标服务器发送DNS请求(即我们配置的本地DNS服务器)。学生的任务是
编写该程序并证明(使用Wireshark)他们的查询请求可以触发目标DNS服务器会发出相应的DNS查询。
该任务对性能的要求不高,因此学生可以使用C语言或Python(使用Scapy)编写此代码。以下提供了
Python的代码示例(其中\texttt{+++}是占位符,学生需要将他们替换为实际的值):


\begin{lstlisting}
Qdsec  = DNSQR(qname='www.example.com')
dns    = DNS(id=0xAAAA, qr=0, qdcount=1, ancount=0, nscount=0,
             arcount=0, qd=Qdsec)

ip  = IP(dst='+++', src='+++')
udp = UDP(dport=+++, sport=+++, chksum=0)
request = ip/udp/dns
\end{lstlisting}
 

\paragraph{Scapy:} 如果你使用Python3,那么在SEED VM中可能没有预先安装Scapy,你需要使用
一下命令安装Python3的Scapy包。


\begin{lstlisting}
$ sudo pip3 install scapy
\end{lstlisting}
 

% -------------------------------------------
% SUBSECTION
% ------------------------------------------- 
\subsection{任务 5: 伪造DNS响应}   

在此任务中,我们需要伪造Kaminsky攻击中的DNS响应。由于我们的攻击目标是\texttt{example.com},
我们需要欺骗从该域的名称服务器返回的响应。学生首先需要找到\texttt{example.com}域的合法名称服务器
的IP地址(值得注意的是这个域名有许多名称服务器)。

学生可以使用Scapy来实现这个任务,以下的代码示例构建了一个DNS响应包,其中包含了问题部分,
回答部分以及一个名称服务器部分。在这段代码中,我们使用\texttt{+++}作为占位符,学生
需要用Kaminsky攻击中所需要的值来替换。学生需要解释为什么选择这些值。

\begin{lstlisting}
name   = '+++'  
domain = '+++'  
ns     = '+++'

Qdsec  = DNSQR(qname=name)
Anssec = DNSRR(rrname=name,   type='A',  rdata='1.2.3.4', ttl=259200)
NSsec  = DNSRR(rrname=domain, type='NS', rdata=ns, ttl=259200)
dns    = DNS(id=0xAAAA, aa=1, rd=1, qr=1,
             qdcount=1, ancount=1, nscount=1, arcount=0,
             qd=Qdsec, an=Anssec, ns=NSsec)

ip    = IP(dst='+++', src='+++')
udp   = UDP(dport=+++, sport=+++, chksum=0)
reply = ip/udp/dns
\end{lstlisting}
 

由于这些响应本身无法进行成功的攻击,为了展示这个任务的效果,学生需要使用Wireshark来
捕获伪造的DNS响应,并证明伪造包是合法的。



% -------------------------------------------
% SUBSECTION
% ------------------------------------------- 
\subsection{任务 6: 进行Kaminsky攻击}   

现在我们将所有东西合在一起进行Kaminsky攻击。在攻击中,我们需要发送许多欺骗的DNS响应,
希望其中有一个可以猜中正确的传输ID并比合法的响应更早到达。因此,发包速度至关重要:我们能
发出越多的数据包,我们的成功率也就越高。如果我们使用Scapy来发送DNS响应包就如在之前的任务那样,
那么成功率将非常低。学生可以使用C语言进行实现,但在C语言中构造DNS数据包并非易事。
因此我们引入了使用C语言和Scapy相结合的混合方法。


通过混合方法,我们首先使用Scapy生成DNS数据包模板,并把模板保存在文件中。
接着我们将该数据包模板加载到C程序中,并对其中某些字段进行一些微小修改,然后发出这个数据包。
我们在实验网站提供了C语言的代码框架 (\texttt{attack.c})。
学生可以对其中标记的区域进行修改,详细的代码解释在之后的指南部分中。



\paragraph{检查DNS缓存:}
为了检查攻击是否成功,我们需要查看{\tt dump.db}文件来检查伪造的DNS响应是否成功被DNS服务器接收。
以下的命令行脚本可以转储DNS缓存,并搜索缓存中是否存在\texttt{attacker}字词(在我们的攻击中,我们
采用\texttt{attacker32.com}作为攻击者的域名,如果学生使用不同的攻击域名,那么需要搜索不同的字词)。
 

\begin{lstlisting}
#!/bin/bash

sudo rndc dumpdb -cache
cat /var/cache/bind/dump.db | grep attacker
\end{lstlisting}
 

% -------------------------------------------
% SUBSECTION
% ------------------------------------------- 
\subsection{任务 7: 攻击结果验证}

如果攻击成功,在本地DNS服务器的缓存中,\texttt{example.com}的{\tt NS}记录应该会改为
\texttt{ns.attacker32.com}。当服务器收到对\texttt{example.com}域内的任何子域名的解析请求时,
它会向\texttt{ns.attacker32.com}发送查询请求,而不是原本合法的名称服务器。


为了验证攻击是否成功,在用户主机上运行以下两条\texttt{dig}命令。在两个响应中,
\texttt{www.example.com}的IP地址应该相同,并且响应应该是在攻击主机的区域文件中配置的内容。


\begin{lstlisting}
// Ask the local DNS server for query
$ dig www.example.com

// Direct request to attacker32 name server
$ dig @ns.attacker32.com www.example.com
\end{lstlisting}
 
请在实验报告中给出你的观察结果(截图),并解释你认为攻击成功的原因。
特别的,当你第一次运行\texttt{dig}命令时,请使用Wireshark来捕获网络流量,
并指出\texttt{dig}命令触发了哪些数据包。根据数据包追踪来证明你的攻击成功了。





% *******************************************
% SECTION
% ******************************************* 
\section{指南} 

为了实现Kaminsky攻击,我们使用Scapy进行数据包欺骗。不幸的是,Python的速度太慢了,Python每秒
生成的数据包太少以至于很难攻击成功。最好的情况是使用C程序,然而这对于许多学生来说颇具挑战,
因为用C构造DNS数据包并非易事。我们开发了一种混合方法,并在课堂上进行了实验。通过这种方法,
可以大大减少学生在编码上花费的时间,因此他们可以将更多的时间用于关注实际攻击的本身。


这个想法是同时利用Scapy和C的优势:Scapy在构建DNS数据包方便比C更方便,但是C速度更快。
因此我们使用Scapy构建伪造的DNS数据包,并将它保存到文件中,接着我们将数据包加载到C程序中。
尽管在Kaminsky攻击过程中,我们需要发送许多不同的DNS数据包,但除了少数字段外,这些数据包几乎相同。
我们可以将Scapy生成的数据包作为基础,找到需要修改地方的偏移量(如传输ID字段),并直接进行修改。
这比在C中创建整个DNS数据包要简单很多。
进行修改之后,我们使用原始的套接字发送这些数据包。有关这种混合方法的详细信息,请参见SEED书~\cite{seedbook}中
``数据包嗅探和伪造''一章。以下的Scapy程序会创建一个简单的DNS响应数据包,并将其保存在文件中。


\begin{lstlisting}[caption={\texttt{generate\_dns\_reply.py}}]
#!/usr/bin/python3
from scapy.all import *

# Construct DNS header and content
name   = 'twysw.example.com'
Qdsec  = DNSQR(qname=name)
Anssec = DNSRR(rrname=name, type='A', rdata='1.1.2.2', ttl=259200)
dns    = DNS(id=0xAAAA, aa=1, rd=0, qr=1, 
             qdcount=1, ancount=1, nscount=0, arcount=0, 
             qd=Qdsec, an=Anssec)

# Construct IP, UDP header and complete data packet
ip  = IP(dst='10.0.2.7', src='1.2.3.4', chksum=0)
udp = UDP(dport=33333, sport=53, chksum=0)
pkt = ip/udp/dns

# Save data package to file
with open('ip.bin', 'wb') as f:
  f.write(bytes(pkt))
\end{lstlisting}

在C程序中,我们从文件\texttt{ip.bin}中读入数据包,并将其用作数据包的模板,在此模板上,
我们可以创建许多类似的数据包,并向本地DNS服务器发送这些欺骗包。对于每个答复,我们修改
三个位置:传输ID和在两个位置(询问部分和回答部分)出现的的\texttt{twysw}。传输ID是一个固定的
位置(从IP数据包开头偏移\texttt{28}),但名称\texttt{twysw}的偏移位置取决于域名的长度。
我们可以使用二进制编辑器,如\texttt{bless},来查看二进制文件\texttt{ip.bin}并找到
\texttt{twysw}的两个偏移量。在我们的数据包中,他们的偏移是\texttt{41} 和 \texttt{64}。


以下的代码片段显示了我们如何修改这些字段。我们将响应中的域名改为\texttt{bbbbb.example.com},
并发出一个伪造的DNS答复(传输ID为\texttt{1000})。
在代码中,变量\texttt{ip}指向IP数据包的起始点。
 

\begin{lstlisting}
  // Modify the domain name of the query field (offset=41)
  memcpy(ip+41, "bbbbb" , 5);

  // Modify the domain name of the answer field (offset=64)
  memcpy(ip+64, "bbbbb" , 5);

  // Modify the transmission ID field (offset=28)
  unsigned short id = 1000;
  unsigned short id_net_order = htons(id);
  memcpy(ip+28, &id_net_order, 2);
\end{lstlisting}



\paragraph{生成随机子域名:} 在Kaminsky攻击中,我们需要生成随机的子域名。
有许多方法可以做到这一点。以下的代码片段展示了如何生成一个5个字符的随机子域名。


\begin{lstlisting}
char a[26]="abcdefghijklmnopqrstuvwxyz";

// Generate a random subdomain with a length of 5
char name[5];
for (int k=0; k<5; k++)  
   name[k] = a[rand() % 26];
\end{lstlisting}
 



% *******************************************
% SECTION
% ******************************************* 
\section{Submission}

\seedsubmission


%%%%%%%%%%%%%%%%%%%%%%%%%%%%%%%%%%%%%%%%%%
\thispagestyle{empty}
\bibliographystyle{plain}
\def\baselinestretch{1}
\bibliography{BibDNS}
%%%%%%%%%%%%%%%%%%%%%%%%%%%%%%%%%%%%%%%%%%



\end{document}
%%%%%%%%%%%%%%%%%%%%%%%%%%%%%%%%%%%%%%%%%%%
%%%%%%%%%%%%%%%%%%%%%%%%%%%%%%%%%%%%%%%%%%%
%%%%%%%%%%%%%%%%%%%%%%%%%%%%%%%%%%%%%%%%%%%
%%%%%%%%%%%%%%%%%%%%%%%%%%%%%%%%%%%%%%%%%%%
%%%%%%%%%%%%%%%%%%%%%%%%%%%%%%%%%%%%%%%%%%%
%%%%%%%%%%%%%%%%%%%%%%%%%%%%%%%%%%%%%%%%%%%
%%%%%%%%%%%%%%%%%%%%%%%%%%%%%%%%%%%%%%%%%%%



