%%%%%%%%%%%%%%%%%%%%%%%%%%%%%%%%%%%%%%%%%%%%%%%%%%%%%%%%%%%%%%%%%%%%%%
%%  Copyright by Wenliang Du.                                       %%
%%  This work is licensed under the Creative Commons                %%
%%  Attribution-NonCommercial-ShareAlike 4.0 International License. %%
%%  To view a copy of this license, visit                           %%
%%  http://creativecommons.org/licenses/by-nc-sa/4.0/.              %%
%%%%%%%%%%%%%%%%%%%%%%%%%%%%%%%%%%%%%%%%%%%%%%%%%%%%%%%%%%%%%%%%%%%%%%

\input{../../common-files/header}
\input{../../common-files/copyright}


\newcommand{\pkiFigs}{./Figs}

\newcommand{\OpenSSL} {\texttt{OpenSSL}\xspace}
\newcommand{\pkiserver}{\texttt{SEEDPKILab2020.com}\xspace}

\lhead{\bfseries SEED Labs -- PKI 实验}


\begin{document}

\begin{center}
{\LARGE 公钥基础设施 (PKI) 实验}
\end{center}

\seedlabcopyright{2018}



% *******************************************
% SECTION
% *******************************************
\section{概述}



公钥密码学是当今安全通信的基础。
但是当通信的一方将其公钥发送给另一方时,会受到中间人的攻击。
根本的问题是,验证公钥的所有权不是一个容易解决的问题。
也就是给定公钥及其声明的所有者信息,我们如何确保公钥确实由声明的所有者拥有?
公钥基础设施(PKI)就是解决此问题的一种方法。


该实验的目的是让学生获得有关PKI的第一手经验。
SEED labs 有一系列关于公钥加密的实验,而这个实验则针对PKI。
通过完成本实验中的任务,学生应该能够更好地理解PKI的工作原理,了解如何使用PKI保护Web以及如何用PKI克服中间人攻击。
此外,学生将能够了解公钥基础设施中信任的根源,以及如果信任根被破坏,将会出现哪些问题。
本实验涵盖以下主题:

\begin{itemize}[noitemsep]
\item 公钥加密
\item 公钥基础设施(PKI)
\item 证书颁发机构(CA)和根 CA
\item X.509 证书和自签名证书
\item Apache, HTTP, 和 HTTPS
\item 中间人攻击
\end{itemize}



\paragraph{阅读材料}
有关PKI的详细信息,请阅读以下内容:

\begin{itemize}
\item Chapter 23 of the SEED Book, \seedbook
\end{itemize}


\paragraph{相关实验}

与本实验相关的一个话题是基于PKI的传输层安全(TLS)。我们有一个单独的TLS实验。
另外,还有一个 \textit{RSA公钥加密和签名实验},其重点是公钥加密的算法部分。


\paragraph{实验环境} \seedenvironment



% *******************************************
% SECTION
% *******************************************
\section{实验内容}


% -------------------------------------------
% SUBSECTION
% -------------------------------------------
\subsection{任务 1: 构建一个证书颁发机构(CA)}

证书颁发机构(CA)是颁发数字证书的一个受信任的实体。
数字证书通过证书指定的主体(Subject)证明公钥的所有权。
有许多商业化的CA被视为根CA。
在撰写本文时,VeriSign是最大的CA。
用户需要付费才能获得这些商业CA颁发的数字证书。


在这个实验中,我们需要创建数字证书,但是我们不会向任何商业CA付费。
我们会自己创建一个根CA,然后使用该CA为其他实体(例如服务器)颁发证书。
在此任务中,我们将自己设为根CA,并为此CA生成一个证书。
与通常由另一个CA签名的其他证书不同,根CA的证书是自签名的。
根CA的证书通常预加载到大多数操作系统,Web浏览器和其他依赖PKI的软件中。
根CA的证书是无条件信任的。


\paragraph{配置文件 {\tt openssl.conf}}
要使用 \OpenSSL 来创建证书,首先需要有一个配置文件。
配置文件的扩展名通常为 {\tt .cnf}。
在 \OpenSSL 的 {\tt ca}、 {\tt req} 和 {\tt x509} 命令中会使用到这个配置文件。
\texttt{openssl.conf} 的说明文档可以用搜索引擎找到,
也可以在 \path{/usr/lib/ssl/openssl.cnf} 中找到一份配置文件。
把这个文件复制到你的当前目录之后,你需要创建几个在配置文件中指定的子目录
(参阅 {\tt [CA\_default]} 一节):


\begin{lstlisting}
   dir             = ./demoCA        # Where everything is kept
   certs           = $dir/certs      # Where the issued certs are kept
   crl_dir         = $dir/crl        # Where the issued crl are kept
   new_certs_dir   = $dir/newcerts   # default place for new certs.
   database        = $dir/index.txt  # database index file.
   serial          = $dir/serial     # The current serial number
\end{lstlisting}

对于 \texttt{index.txt} 文件, 创建一个空文件即可。
对于 \texttt{serial} ,放一个字符串格式的数(例如 1000)在文件中。
当你设置好了配置文件 \texttt{openssl.cnf} 之后就可以创建和颁发证书了。


\paragraph{证书颁发机构(CA)}
如前所述,我们需要为我们的CA生成一个自签名证书。
这意味着该CA是完全受信任的,并且其证书将用作根证书。
你可以运行以下命令为CA生成自签名证书:

\begin{lstlisting}[backgroundcolor=]
$ openssl req -new -x509 -keyout ca.key -out ca.crt -config openssl.cnf
\end{lstlisting}

系统将提示你输入一些信息和密码。
不要丢失此密码,因为每次要使用此CA为其他人签名证书时,都必须输入密码。
它还将要求填写一些信息,例如“国家名称”,“通用名称”等。
命令的输出存储在两个文件中:{\tt ca.key} 和 {\tt ca.crt}。
文件{\tt ca.key}包含CA的私钥,而{\tt ca.crt}包含公钥证书。



% -------------------------------------------
% SUBSECTION
% -------------------------------------------
\subsection{任务 2: 为 \pkiserver 创建一个证书}

现在我们已经成为了一个根CA,准备给我们的客户签发数字证书。
我们的第一个客户是一个名为 \pkiserver 的公司。
为了使该公司从CA获得数字证书,需要经历三个步骤。

\paragraph{第 1 步: 生成公私钥对}
公司首先需要创建自己的公私钥对。
我们可以运行以下命令来生成RSA密钥对(私钥和公钥)。
你还需要提供密码来加密私钥(在命令选项指定了使用AES-128加密算法)。
密钥将存储在文件 \texttt{server.key} 中:

\begin{lstlisting}[backgroundcolor=]
$ openssl genrsa -aes128 -out server.key 1024
\end{lstlisting}

\texttt{server.key} 是一个经过编码(和加密)的文本文件,因此你无法看到实际的内容,例如模数,私有指数等。
要查看这些内容,可以运行以下命令:

\begin{lstlisting}[backgroundcolor=]
$ openssl rsa -in server.key -text
\end{lstlisting}



\paragraph{第 2 步: 生成一个证书签发请求 (CSR).}
公司拥有密钥文件后,应当生成一个证书签名请求(CSR)。
该证书签名请求会包括公司的公钥。
CSR将被发送到CA,CA将为密钥生成证书(在确保CSR中的身份信息与服务器的真实身份匹配之后)。
请使用 \pkiserver 作为证书请求的通用名称(Common Name)。

\begin{lstlisting}[backgroundcolor=]
$ openssl req -new -key server.key -out server.csr -config openssl.cnf
\end{lstlisting}


注意到,以上命令与我们在为CA创建自签名证书时使用的命令非常相似。
它们唯一的区别是{\tt -x509}选项。
没有这个选项,该命令将生成一个证书签发请求;
加上这个选项,该命令将生成一个自签名证书。

\paragraph{第 3 步: 生成证书}
CSR文件需要具有CA的签名才能形成证书。
在现实世界中,通常将CSR文件发送到受信任的CA进行签名。
在本实验中,我们将使用我们自己的受信任CA生成证书。
以下命令使用CA的 {\tt ca.crt} 和 {\tt ca.key} ,将证书签名请求( {\tt server.csr} )转换为X509证书({\tt server.crt}):

\begin{lstlisting}[backgroundcolor=]
$ openssl ca -in server.csr -out server.crt -cert ca.crt -keyfile ca.key \
             -config openssl.cnf
\end{lstlisting}

如果 \OpenSSL 拒绝生成证书,那么你的请求中的名称很可能与 CA 的名称不匹配。
匹配规则在配置文件中指定(请参见 \texttt{[policy\_match]} 部分)。
你可以更改请求名称以符合策略,也可以更改策略。
配置文件还包含另一个策略(称为 \texttt {policy\_anything}),该策略的限制较少。
你可以通过更改以下行来选择该策略:

\begin{lstlisting}[backgroundcolor=]
   "policy = policy_match"  change to "policy = policy_anything".
\end{lstlisting}



% -------------------------------------------
% SUBSECTION
% -------------------------------------------
\subsection{任务 3: 部署证书到一个 HTTPS Web 服务器上}

在本实验中,我们将探索网站是如何用证书来保护 Web 的。
我们将使用 \OpenSSL 的内置Web服务器建立一个HTTPS网站。


\paragraph{第 1 步: 配置DNS}
我们选择 {\pkiserver} 作为网站的名字.
为了让我们的计算机能识别这个名字,我们需要在 \texttt{/etc/hosts} 中添加下面这一项。
这一项把域名 {\pkiserver} 映射到 localhost (也就是 127.0.0.1):


\begin{lstlisting}[backgroundcolor=]
   127.0.0.1  SEEDPKILab2018.com
\end{lstlisting}


\paragraph{第 2 步: 配置 Web 服务器}
让我们使用上一个任务中生成的证书启动一个简单的 Web 服务器。
\OpenSSL 允许我们使用 \texttt{s\_server} 命令启动简单的 Web 服务器:

\begin{lstlisting}
  # Combine the secret key and certificate into one file
  % cp server.key server.pem
  % cat server.crt >> server.pem

  # Launch the web server using server.pem
  % openssl s_server -cert server.pem -www
\end{lstlisting}

默认情况下,服务器会监听 {\tt 4433} 端口。
你可以使用 {\tt -accept} 选项更改端口。
现在,使用以下 URL 访问服务器:\url{https://SEEDPKILab2018.com:4433/}。
你将很有可能在浏览器中看到一条错误消息。
在 Firefox 中,你会看到类似这样的消息:
{\em ``seedpkilab2018.com:4433 uses an invalid security certificate. The certificate is not trusted because the issuer certificate is unknown''.}


\paragraph{第 3 步: 使浏览器接受我们的证书}
如果我们的证书是由 VeriSign 分配的,我们将不会收到这样的错误消息,因为 VeriSign 的证书很有可能已经预先加载到 Firefox 的证书库中。
不幸的是,\pkiserver 证书是由我们自己的CA签名的(即使用 {\tt ca.crt}),并且 Firefox 无法识别该CA。
有两种方法可以使 Firefox 接受我们的CA自签名证书。

\begin{itemize}

\item 我们可以要求 Mozilla 在其 Firefox 软件中加入我们的CA证书,这样使用 Firefox 的每个人都可以识别我们的CA。
这就是真正的CA(例如VeriSign)将其证书添加到Firefox中的方式。
不幸的是,我们自己的CA没有足够大的市场供Mozilla包含我们的证书,因此我们不会采用这种方法。

\item {\bf 将 {\tt ca.crt} 导入 Firefox :}
通过点击以下菜单,我们可以将CA的证书手动添加到 Firefox 浏览器:

\begin{lstlisting}[backgroundcolor=]
   Edit -> Preference -> Privacy & Security -> View Certificates.
\end{lstlisting}

你会看到Firefox已经接受的证书列表。
从这里,我们可以 ``导入'' 我们自己的证书。
请导入 {\tt ca.crt} ,然后选择 ``信任此CA可以标识网站'' 。
你会看到我们的CA证书现在位于Firefox的已接受证书列表中。
\end{itemize}


\paragraph{第 4 步: 测试我们的 HTTPS 网站}
现在访问 \url{https://SEEDPKILab2018.com:4433}。
请描述和解释你观察到的现象,并完成以下任务:

\begin{enumerate}
\item 修改 {\tt server.pem} 中的一个字节,重新启动服务器并刷新页面。
      你观察到了什么?
      确保你之后能回复原始的 {\tt server.pem}。
      注意:如果 {\tt server.pem} 中某些特定的部分被破坏,服务器可能无法重启。
      在这种情况下,选择另一个位置修改。


\item 既然 \pkiserver 指向了 localhost,访问 \url{https://localhost:4433} 将会连接到同一个 Web 服务器。
      请这样尝试一下,描述并解释你观察到的现象。
\end{enumerate}



% -------------------------------------------
% SUBSECTION
% -------------------------------------------
\subsection{任务 4: 在基于Apache的HTTPS网站中部署证书}


使用 \openssl 的 \texttt{s\_server} 命令设置 HTTPS 服务器主要用于调试和演示。
在本实验中,我们基于 Apache 建立了一个真实的 HTTPS Web 服务器。
我们的VM中已经安装了Apache服务器,它支持HTTPS协议。
要创建HTTPS网站,我们只需要配置Apache服务器,让它知道从哪里获取私钥和证书。
我们在下面给出一个示例,演示如何为网站 \url{www.example.com} 启用HTTPS。
你的任务是使用从先前任务生成的证书对 {\pkiserver} 执行相同的操作。

一个Apache服务器可以同时托管多个网站。
它需要知道网站文件的存储目录。
这是通过位于 \url{/etc/apache2/sites-available} 目录中的 \texttt{VirtualHost} 文件完成的。
要添加HTTP网站,我们将 \texttt{VirtualHost} 条目添加到文件 \texttt{000-default.conf}。
请参见以下示例。

\begin{lstlisting}
<VirtualHost *:80>
    ServerName one.example.com
    DocumentRoot /var/www/Example_One
    DirectoryIndex index.html
</VirtualHost>
\end{lstlisting}

要添加一个 HTTPS 网站,我们需要在相同文件夹里的 \texttt{default-ssl.conf} 文件中添加一个 \texttt{VirtualHost} 条目。

\begin{lstlisting}
<VirtualHost *:443>
    ServerName two.example.com
    DocumentRoot /var/www/Example_Two
    DirectoryIndex index.html

    SSLEngine On
    SSLCertificateFile      /etc/apache2/ssl/example_cert.pem  (*@\ding{192}@*)
    SSLCertificateKeyFile   /etc/apache2/ssl/example_key.pem   (*@\ding{193}@*)
</VirtualHost>
\end{lstlisting}

\texttt{ServerName} 指明了网站的名字,而 \texttt{DocumentRoot} 指明了存放网站文件的位置。
上面的例子配置了 HTTPS 网站 \url{https://two.example.com}  (端口 \texttt{443}
是 HTTPS 的默认端口)。
在设置中,我们需要告诉 Apache 服务器证书 (\ding{192}) 和私钥 (\ding{193}) 存放在了哪里。


在 \texttt{default-ssl.conf} 文件被修改以后,我们需要运行一系列命令来启用 SSL。
Apache 将会要求我们输入用于加密私钥的密码。
当这些都正确和之后以后,我们可以浏览这个网站。
浏览器与服务器之间所有的流量都会被加密。

\begin{lstlisting}
 // Test the Apache configuration file for errors
 $ sudo apachectl configtest

 // Enable the SSL module
 $ sudo a2enmod ssl

 // Enable the site we have just edited
 $ sudo a2ensite default-ssl

 // Restart Apache
 $ sudo service apache2 restart
\end{lstlisting}



请使用参考以上示例为 \pkiserver 设置 HTTPS 服务器。
请描述你的操作步骤,添加到 Apache 的配置文件中的内容以及最终结果的屏幕截图,以展示你可以成功浏览HTTPS站点。



% -------------------------------------------
% SUBSECTION
% -------------------------------------------
\subsection{任务 5: 发起中间人攻击}

在此任务中,我们将展示 PKI 如何抵御中间人(MITM)攻击。
图 \ref{pki:fig:mitm} 描述了MITM攻击的工作方式。
假设 Alice 想通过HTTPS协议访问 \texttt{example.com}。
她需要从 \texttt{example.com} 服务器获取公钥; Alice 将生成一个秘密,并使用服务器的公钥对该秘密进行加密,然后将其发送到服务器。
如果攻击者可以拦截 Alice 与服务器之间的通信,那么攻击者可以用其自己的公钥替换服务器的公钥。
这样 Alice 的秘密实际上是使用攻击者的公钥加密的,因此攻击者将能够读取该秘密。
攻击者可以使用服务器的公钥将秘密转发给服务器。
因为该秘密将用于加密 Alice 和服务器之间的通信,因此攻击者可以解密加密的通信。


\begin{figure}[htb]
   \begin{center}
      \includegraphics[width=0.8\textwidth]{\pkiFigs/mitm.pdf}
   \end{center}
   \caption{中间人攻击}
   \label{pki:fig:mitm}
\end{figure}



该任务的目的是帮助学生了解 PKI 如何抵御此类MITM攻击。
在任务中,我们将模拟一次MITM攻击,并了解PKI如何精确地防御。
我们将首先选择一个目标网站。
在本文档中,我们使用 \texttt{example.com} 作为目标网站。
但在实际任务中,为了使其更有意义,学生可以选择一个受欢迎的网站,例如银行网站或者社交网站。


\paragraph{第 1 步: 启动一个恶意网站}
在任务4中,我们已经为 \pkiserver 建立了HTTPS网站。
我们将使用同一台 Apache 服务器来模拟 \texttt{example.com} (或学生选择的站点)。
为此,我们将按照任务4中的说明向Apache的SSL配置文件中添加 \texttt{VirtualHost} 条目, \texttt{ServerName} 应该为 \texttt{example.com} ,但其余配置可以与任务4中使用的相同。
我们的目标如下:当用户尝试访问 \texttt{example.com} 时,我们将使该用户进入我们的服务器,该服务器托管一个伪造的 \texttt{example.com}。
如果这是一个社交网络网站,则假网站可以显示类似于目标网站中的登录页面。
如果用户无法分辨出区别,则可以在伪造的网页中键入其帐户凭据,从而使攻击者获得凭据。


\paragraph{第 2 步: 成为中间人}
有几种方法可以使用户的 HTTPS 请求进入我们的 Web 服务器。
一种方法是路由攻击,使用户的 HTTPS 请求被路由到我们的 Web 服务器。
另一种方法是 DNS 攻击,当受害者的计算机尝试找出目标 Web 服务器的 IP 地址时,它将获取我们 Web 服务器的 IP 地址。
在此任务中,我们使用 DNS 攻击。 无需发动实际的 DNS 缓存中毒攻击,我们只需修改受害者机器的 \texttt{/etc/hosts} 文件,以模拟 DNS 缓存存储攻击的结果(应使用恶意服务器的实际IP地址替换 \texttt{IP\_Address})。



\begin{lstlisting}[backgroundcolor=]
   <IP_Address>  example.com
\end{lstlisting}


\paragraph{第 3 步: 浏览目标网站}
完成所有设置后,现在访问目标真实网站,并查看您的浏览器会说些什么。
请解释你观察到的现象。




% -------------------------------------------
% SUBSECTION
% -------------------------------------------
\subsection{任务 6: 使用一个被攻陷的 CA 发起中间人攻击}

不幸的是,我们在任务1中创建的根CA被攻击者攻破,并且其私钥被盗。
因此,攻击者可以使用此CA的私钥生成任意证书。
在此任务中,我们将看到这种破坏的结果。



请设计一个实验,以表明攻击者可以在任何HTTPS网站上成功发起MITM攻击。
你可以使用在任务5中创建的相同设置,但是这次,你需要证明MITM攻击是成功的。
即当受害人试图访问网站时,浏览器不会有起任何怀疑,而是落入MITM攻击者的虚假网站。




% *******************************************
% SECTION
% *******************************************
\section{提交}

\seedsubmission


%%%%%%%%%%%%%%%%%%%%%%%%%%%%%%%%%%%%%%%%%%%%%%%%%%%%%%
\end{document}
%%%%%%%%%%%%%%%%%%%%%%%%%%%%%%%%%%%%%%%%%%%%%%%%%%%%%%

