%%%%%%%%%%%%%%%%%%%%%%%%%%%%%%%%%%%%%%%%%%%%%%%%%%%%%%%%%%%%%%%%%%%%%%
%%  Copyright by Wenliang Du.                                       %%
%%  This work is licensed under the Creative Commons                %%
%%  Attribution-NonCommercial-ShareAlike 4.0 International License. %%
%%  To view a copy of this license, visit                           %%
%%  http://creativecommons.org/licenses/by-nc-sa/4.0/.              %%
%%%%%%%%%%%%%%%%%%%%%%%%%%%%%%%%%%%%%%%%%%%%%%%%%%%%%%%%%%%%%%%%%%%%%%

\newcommand{\commonfolder}{../../common-files}

\input{\commonfolder/header}
\input{\commonfolder/copyright}

\newcommand{\pkiFigs}{./Figs}

\newcommand{\OpenSSL} {\texttt{OpenSSL}\xspace}
\newcommand{\pkiserver}{\texttt{bank32.com}\xspace}

\lhead{\bfseries SEED Labs -- PKI 实验}


\begin{document}

\begin{center}
{\LARGE 公钥基础设施 (PKI) 实验}
\end{center}

\seedlabcopyright{2018}



% *******************************************
% SECTION
% *******************************************
\section{概述}



公钥密码学是当今安全通信的基础。
但是当通信的一方将其公钥发送给另一方时,会受到中间人的攻击。
根本的问题是,验证公钥的所有权不是一个容易解决的问题。
也就是给定公钥及其声明的所有者信息,我们如何确保公钥确实由声明的所有者拥有?
公钥基础设施(PKI)就是解决此问题的一种方法。


该实验的目的是让学生获得有关 PKI 的第一手经验。
SEED labs 有一系列关于公钥加密的实验,而这个实验则针对 PKI 。
通过完成本实验中的任务,学生应该能够更好地理解 PKI 的工作原理,
了解如何使用 PKI 保护 Web 以及如何用 PKI 抵御中间人攻击。
此外,学生将能够了解公钥基础设施中信任的根源,以及如果信任根被破坏,将会出现哪些问题。
本实验涵盖以下主题:

\begin{itemize}[noitemsep]
   \item 公钥加密与公钥基础设施(PKI)
   \item 证书颁发机构(CA)、X.509 证书和根 CA
   \item Apache, HTTP, 和 HTTPS
   \item 中间人攻击
\end{itemize}



\paragraph{阅读材料}
有关PKI的详细信息,请阅读以下内容:

\begin{itemize}
\item Chapter 23 of the SEED Book, \seedbook
\end{itemize}


\paragraph{相关实验}

与本实验相关的一个话题是基于PKI的传输层安全(TLS)。我们有一个单独的TLS实验。
另外,还有一个 \textit{RSA公钥加密和签名实验},其重点是公钥加密的算法部分。


\paragraph{实验环境} \seedenvironmentB


% *******************************************
% SECTION
% *******************************************
\section{实验环境}

在本实验中,我们将生成公钥证书,然后使用它们来保护 Web 服务器。
证书生成任务将在虚拟机上执行,但是我们将使用容器托管 Web 服务器。

\paragraph{容器安装及其命令}
%%%%%%%%%%%%%%%%%%%%%%%%%%%%%%%%%%%%%%%%%%%%
请从实验的网站下载 \texttt{Labsetup.zip} 文件到你的虚拟机中,解压它,
进入 \texttt{Labsetup} 文件夹,然后用 \texttt{docker-compose.yml} 文件安装实验环境。
对这个文件及其包含的所有 \texttt{Dockerfile} 文件中的内容的详细解释都可以
在链接到本实验网站的用户手册 \footnote{
    如果你在部署容器的过程中发现从官方源下载容器镜像非常慢,
    可以参考手册中的说明使用当地的镜像服务器
} 中找到。
如果这是您第一次使用容器设置 SEED 实验环境,那么阅读用户手册非常重要。

在下面,我们列出了一些与 Docker 和 Compose 相关的常用命令。
由于我们将非常频繁地使用这些命令,因此我们在 \texttt{.bashrc} 文件
(在我们提供的 SEEDUbuntu 20.04 虚拟机中)中为它们创建了别名。


\begin{lstlisting}
$ docker-compose build  # Build the container image
$ docker-compose up     # Start the container
$ docker-compose down   # Shut down the container

// Aliases for the Compose commands above
$ dcbuild       # Alias for: docker-compose build
$ dcup          # Alias for: docker-compose up
$ dcdown        # Alias for: docker-compose down
\end{lstlisting}


所有容器都在后台运行。
要在容器上运行命令,我们通常需要获得容器里的 shell 。
首先需要使用 \texttt{docker ps} 命令找出容器的 ID ,
然后使用 \texttt{docker exec} 在该容器上启动 shell 。
我们已经在 \texttt{.bashrc} 文件中为这两个命令创建了别名。

\begin{lstlisting}
$ dockps        // Alias for: docker ps --format "{{.ID}}  {{.Names}}"
$ docksh <id>   // Alias for: docker exec -it <id> /bin/bash

// The following example shows how to get a shell inside hostC
$ dockps
b1004832e275  hostA-10.9.0.5
0af4ea7a3e2e  hostB-10.9.0.6
9652715c8e0a  hostC-10.9.0.7

$ docksh 96
root@9652715c8e0a:/#

// Note: If a docker command requires a container ID, you do not need to
//       type the entire ID string. Typing the first few characters will
//       be sufficient, as long as they are unique among all the containers.
\end{lstlisting}


如果你在设置实验环境时遇到问题,可以尝试从手册的``Miscellaneous Problems''部分中寻找解决方案。

%%%%%%%%%%%%%%%%%%%%%%%%%%%%%%%%%%%%%%%%%%%%



\paragraph{DNS 设置}

在本文档中,我们以 \texttt{www.bank32.com} 为例说明如何部署具有此域名的 HTTPS Web 服务器。
学生需要使用其他名称完成实验。
除非教师额外指定名称,否则学生应在服务器名称中包括其姓氏和实验年份。
例如,王小二在 2020 年进行了此实验,
则服务器名称应为 \texttt{www.wang2020.com}。
你不需要拥有此域名,你只需要通过在 \texttt{/etc/hosts}
中添加以下条目即可将此名称映射到容器的IP地址
(第一个条目是必需的,否则,本实验说明中的示例将会失效):

\begin{lstlisting}
10.9.0.80   www.bank32.com
10.9.0.80   www.wang2020.com
\end{lstlisting}


% *******************************************
% SECTION
% *******************************************
\section{实验内容}


% -------------------------------------------
% SUBSECTION
% -------------------------------------------
\subsection{任务 1: 构建一个证书颁发机构(CA)}

证书颁发机构(CA)是颁发数字证书的一个受信任的实体。
数字证书通过证书指定的主体(Subject)证明公钥的所有权。
有许多商业化的CA被视为根CA。
在撰写本文时,VeriSign是最大的CA。
用户需要付费才能获得这些商业CA颁发的数字证书。


在这个实验中,我们需要创建数字证书,但是我们不会向任何商业CA付费。
我们会自己创建一个根CA,然后使用该CA为其他实体(例如服务器)颁发证书。
在此任务中,我们将自己设为根CA,并为此CA生成一个证书。
与通常由另一个CA签名的其他证书不同,根CA的证书是自签名的。
根CA的证书通常预加载到大多数操作系统,Web浏览器和其他依赖PKI的软件中。
根CA的证书是无条件信任的。


\paragraph{配置文件 {\tt openssl.conf}}
要使用 \OpenSSL 来创建证书,首先需要有一个配置文件。
配置文件的扩展名通常为 {\tt .cnf}。
在 \OpenSSL 的 {\tt ca}、 {\tt req} 和 {\tt x509} 命令中会使用到这个配置文件。
\texttt{openssl.conf} 的说明文档可以网络上找到,
\OpenSSL 默认会使用 \path{/usr/lib/ssl/openssl.cnf} 作为配置文件。
由于我们需要对这个文件做出一些改动,我们把它复制到当前目录,
并指定 \OpenSSL 使用这个副本。

配置文件中的 \texttt{[CA\_default]} 一节展示了我们需要准备的默认设置。
我们需要创建几个子目录。
请去掉 \texttt{unique\_subject} 一行的注释,以允许创建有相同主体的多张证书,
因为在这个实验中我们很可能遇到这种情况。

\begin{lstlisting}[caption={Default CA setting}]
[ CA_default ]
dir             = ./demoCA         # Where everything is kept
certs           = $dir/certs       # Where the issued certs are kept
crl_dir         = $dir/crl         # Where the issued crl are kept
database        = $dir/index.txt   # database index file.
#unique_subject = no               # Set to 'no' to allow creation of
                                    # several certs with same subject.
new_certs_dir   = $dir/newcerts    # default place for new certs.
serial          = $dir/serial      # The current serial number
\end{lstlisting}


对于 \texttt{index.txt} 文件, 创建一个空文件即可。
对于 \texttt{serial} ,放一个字符串格式的数(例如 1000)在文件中。
当你设置好了配置文件 \texttt{openssl.cnf} 之后就可以创建和颁发证书了。


\paragraph{证书颁发机构(CA)}
如前所述,我们需要为我们的CA生成一个自签名证书。
这意味着该CA是完全受信任的,并且其证书将用作根证书。
你可以运行以下命令为CA生成自签名证书:

\begin{lstlisting}
openssl req -x509 -newkey rsa:4096 -sha256 -days 3650 \
            -keyout ca.key -out ca.crt
\end{lstlisting}

系统将提示你输入一个密码。
不要丢失此密码,因为每次要使用此CA为其他人签名证书时,都必须输入密码。
它还将要求填写证书主体信息,例如“国家名称”“通用名称”等。
命令的输出存储在两个文件中:{\tt ca.key} 和 {\tt ca.crt}。
文件{\tt ca.key}包含CA的私钥,而{\tt ca.crt}包含公钥证书。


你也可以在命令行中指定主体信息和密码,这样就不会提示你输入任何其他信息。
在以下命令中,我们使用 \texttt{-subj} 设置主体信息;
使用 \texttt{-passout pass:dees} 将密码设置为 \texttt{dees} 。

\begin{lstlisting}
openssl req -x509 -newkey rsa:4096 -sha256 -days 3650 \
            -keyout ca.key -out ca.crt  \
            -subj "/CN=www.modelCA.com/O=Model CA LTD./C=US" \
            -passout pass:dees
\end{lstlisting}


我们可以使用以下命令查看 X509 证书和 RSA 密钥的解码内容
( \texttt{-text} 表示将内容解码为纯文本; \texttt{-noout} 表示不打印出编码版本):

\begin{lstlisting}
openssl x509 -in ca.crt -text -noout
openssl rsa  -in ca.key -text -noout
\end{lstlisting}

请运行以上命令,并从输出中找出以下内容:

\begin{itemize}[noitemsep]
   \item 证书中的哪一部分说明了这是一个 CA 的证书?
   \item 证书中的哪一部分说明了这是一个自签名证书?
   \item 在 RSA 算法中,我们有公开指数 $e$ 、私有指数 $d$、模数 $n$
         以及两个秘密的数 $p$ 和 $q$ 使得 $n = pq$ 。
         请从你的证书和密钥文件中找出这些元素的值。
\end{itemize}


% -------------------------------------------
% SUBSECTION
% -------------------------------------------
\subsection{任务 2: 为你的 Web 服务器生成证书请求}

生成证书签名请求(CSR)的命令与在创建 CA 自签名证书时使用的命令非常相似,
唯一的区别是是否带有 \texttt{-x509} 选项。
没有这个选项,该命令将生成一个证书签发请求;
加上这个选项,该命令将生成一个自签名证书。
以下命令为 \texttt{www.bank32.com} 生成 CSR (您应该使用自己的服务器名称):

\begin{lstlisting}
openssl req -newkey rsa:2048 -sha256  \
            -keyout server.key   -out server.csr  \
            -subj "/CN=www.bank32.com/O=Bank32 Inc./C=US" \
            -passout pass:dees
\end{lstlisting}

该命令将生成一对公私钥对,然后使用公钥创建证书签名请求。
我们可以使用以下命令查看 CSR 和私钥文件的解码内容:

\begin{lstlisting}
openssl req -in server.csr -text -noout
openssl rsa -in server.key -text -noout
\end{lstlisting}

\paragraph{添加备用名称(Alternative Name)}
许多网站都有不同的 URL 。
例如, \texttt{www.example.com} , \texttt{example.com} , \texttt{example.net}
和 \texttt{example.org} 都指向同一 Web 服务器。
由于浏览器实施了主机名匹配策略,因此证书中的公用名(Common Name)必须与服务器的主机名匹配,
否则浏览器将拒绝与服务器通信。

为了使证书具有多个名称, X.509 规范定义了一个可以附加到证书的扩展,名为主体备用名称(SAN)。
使用 SAN 扩展名,可以在证书的 \texttt{subjectAltName} 字段中指定多个主机名。

要使用此类字段生成证书签名请求,我们可以将所有必要的信息放在配置文件中或命令行中。
我们在本任务中使用命令行的方法(在 TLS 实验中会使用配置文件的方法)。
我们可以在 \texttt{openssl req} 命令中添加以下选项。
应当注意, \texttt{subjectAltName} 扩展字段还必须包括通用名称字段中的主机名。
否则,通用名称将不会被接受为有效名称。

\begin{lstlisting}
-addext "subjectAltName = DNS:www.bank32.com,  \
                          DNS:www.bank32A.com, \
                          DNS:www.bank32B.com"
\end{lstlisting}

请在你的证书签名请求中添加两个备用名称。
在后面的任务中将会用到它们。


% -------------------------------------------
% SUBSECTION
% -------------------------------------------
\subsection{任务 3: 为你的服务器生成证书}

CSR文件需要具有CA的签名才能形成证书。
在现实世界中,通常将CSR文件发送到受信任的CA进行签名。
在本实验中,我们将使用我们自己的受信任CA生成证书。
以下命令使用CA的 {\tt ca.crt} 和 {\tt ca.key} ,
将证书签名请求( {\tt server.csr} )转换为X509证书({\tt server.crt}):

\begin{lstlisting}
openssl ca -config myCA_openssl.cnf -policy policy_anything \
            -md sha256 -days 3650 \
            -in server.csr -out server.crt -batch \
            -cert ca.crt -keyfile ca.key
\end{lstlisting}

在上面的命令中,\texttt{myCA\_openssl.cnf} 是我们从 \path{/usr/lib/ssl/openssl.cnf}
复制的配置文件(我们在任务 1 中对此文件进行了更改)。
我们使用配置文件中定义的 \texttt{policy\_anything} 策略,这不是默认策略。
默认策略有更多限制,要求请求中的某些主体信息必须与 CA 证书中的主体信息匹配。
命令中使用的策略不强制执行任何匹配规则。

\paragraph{复制扩展域}
出于安全原因,\texttt{openssl.cnf} 中的默认设置
不允许 \texttt{openssl ca} 命令将扩展字段从请求复制到最终证书。
为此,我们可以在配置文件的副本中,取消以下行的注释:

\begin{lstlisting}
# Extension copying option: use with caution.
copy_extensions = copy
\end{lstlisting}

签署证书后,请使用以下命令输出证书的解码内容,并检查是否包含备用名称。

\begin{lstlisting}
openssl x509 -in server.crt -text -noout
\end{lstlisting}



% -------------------------------------------
% SUBSECTION
% -------------------------------------------
\subsection{任务 4: 在基于Apache的HTTPS网站中部署证书}

在此任务中,我们将看到网站如何使用公钥证书来保护 Web 浏览。
我们将建立一个基于 Apache 的 HTTPS 网站。
我们的容器中已经安装了 Apache 服务器,它支持 HTTPS 协议。
要搭建 HTTPS 网站,我们只需要配置 Apache 服务器,让它知道从哪里获取私钥和证书。
在容器内部,我们已经为 \pkiserver 设置了 HTTPS 站点。
学生可以按照以下示例来设置自己的 HTTPS 网站。

一个Apache服务器可以同时托管多个网站。
它需要知道网站文件的存储目录。
这是通过位于 \url{/etc/apache2/sites-available} 目录中的 \texttt{VirtualHost} 文件完成的。
在我们的容器中,我们有一个名为 \texttt{bank32\_apache\_ssl.conf} 的文件,其中包含以下条目:

\begin{lstlisting}
<VirtualHost *:443>
   DocumentRoot /var/www/bank32
   ServerName www.bank32.com
   ServerAlias www.bank32A.com
   ServerAlias www.bank32B.com
   DirectoryIndex index.html
   SSLEngine On
   SSLCertificateFile    /certs/bank32.crt    (*@\ding{192}@*)
   SSLCertificateKeyFile /certs/bank32.key    (*@\ding{193}@*)
</VirtualHost>
\end{lstlisting}

上面的示例设置了 HTTPS 站点 \url{https://www.bank32.com}
(端口 \texttt{443} 是默认的 HTTPS 端口)。
\texttt{ServerName} 条目指定网站的名称,
而 \texttt{DocumentRoot} 条目指定网站文件的存储位置。
使用 \texttt{ServerAlias} 条目,我们允许网站使用不同的名称。
你也应该提供两个别名条目。


我们还需要告诉 Apache 服务器证书(\ding{192})和私钥(\ding{193})的存储位置。
在 \texttt{Dockerfile} 中,我们已经包含了
用于将证书和密钥复制到容器的 \texttt{/certs}文件夹的命令。

为了使该网站正常工作,我们需要启用 Apache 的 \texttt{ssl} 模块,然后启用该网站。
可以使用以下命令完成操作,这些命令在构建容器时已经执行。

\begin{lstlisting}
# a2enmod ssl                 // Enable the SSL module
# a2ensite bank32_apache_ssl  // Enable the sites described in this file
\end{lstlisting}



\paragraph{启动 Apache 服务器}
在容器中 Apache 服务器不会自动启动。
我们需要在容器里运行以下命令来启动服务器(我们还列出了一些相关命令):

\begin{lstlisting}
// Start the server
# service apache2 start

// Stop the server
# service apache stop

// Restart a server
# service apache restart
\end{lstlisting}

Apache 启动时,需要为每个 HTTPS 站点加载私钥。
我们的私钥已加密,因此 Apache 会要求我们输入密码进行解密。
在容器内,用于 \texttt{bank32} 的密码为 \texttt{dees} 。
如果一切设置正确,我们就可以浏览该网站,并且浏览器和服务器之间的所有流量都将被加密。

请使用以上示例作为指导,为您的网站设置 HTTPS 服务器。
请描述你的操作步骤、添加到 Apache 的配置文件中的内容以及最终结果的屏幕快照,
以展示你可以成功浏览 HTTPS 站点。

\paragraph{浏览网站}
现在,用浏览器访问你的 Web 服务
(注意:你应该在 URL 的开头加上 \texttt{https},而不要使用 \texttt{http})。
请描述你观察到的现象并做出解释。
您很可能无法成功,这是因为……(此处省略了原因,学生应在实验报告中做出解释)。
请解决此问题,并证明你可以成功访问 HTTPS 网站。

下面,我们说明如何将证书加载到 Firefox 。
学生需要自己弄清楚为什么以及应该加载什么证书。
要将证书手动添加到 Firefox 浏览器,请在地址栏中键入以下URL,
然后单击页面上的 \texttt{View Certificates} 按钮(滚动到底部)。

\begin{lstlisting}
about:preferences#privacy
\end{lstlisting}

在 \texttt{Authorities} 标签中,你将看到已被 Firefox 接受的证书列表。
在这里,我们可以导入我们自己的证书。
选择证书文件后,请选择 ``Trust this CA to identify websites'' 选项。
你会看到我们的证书现在在 Firefox 接受的证书列表中。



% -------------------------------------------
% SUBSECTION
% -------------------------------------------
\subsection{任务 5: 发起中间人攻击}

在此任务中,我们将展示 PKI 如何抵御中间人(MITM)攻击。
图 \ref{pki:fig:mitm} 描述了MITM攻击的工作方式。
假设 Alice 想通过HTTPS协议访问 \texttt{example.com}。
她需要从 \texttt{example.com} 服务器获取公钥; Alice 将生成一个秘密,并使用服务器的公钥对该秘密进行加密,然后将其发送到服务器。
如果攻击者可以拦截 Alice 与服务器之间的通信,那么攻击者可以用其自己的公钥替换服务器的公钥。
这样 Alice 的秘密实际上是使用攻击者的公钥加密的,因此攻击者将能够读取该秘密。
攻击者可以使用服务器的公钥将秘密转发给服务器。
因为该秘密将用于加密 Alice 和服务器之间的通信,因此攻击者可以解密加密的通信。


\begin{figure}[htb]
   \begin{center}
      \includegraphics[width=0.8\textwidth]{\pkiFigs/mitm.pdf}
   \end{center}
   \caption{中间人攻击}
   \label{pki:fig:mitm}
\end{figure}


该任务的目的是帮助学生了解 PKI 如何抵御此类MITM攻击。
在任务中,我们将模拟一次MITM攻击,并了解PKI如何精确地防御。
我们将首先选择一个目标网站。
在本文档中,我们使用 \texttt{example.com} 作为目标网站。
但在实际任务中,为了使其更有意义,学生可以选择一个受欢迎的网站,例如银行网站或者社交网站。


\paragraph{第 1 步: 启动一个恶意网站}
在任务4中,我们已经为 \pkiserver 建立了HTTPS网站。
我们将使用同一台 Apache 服务器来模拟 \texttt{www.example.com} (或学生选择的站点)。
为此,我们将按照任务4中的说明向Apache的SSL配置文件中添加 \texttt{VirtualHost} 条目,
\texttt{ServerName} 应该为 \texttt{example.com} ,
但其余配置可以与任务4中使用的相同。
显然在真实世界中你不可能获得一个 \texttt{www.example.com} 的合法证书,
所以我们将会使用与我们自己的服务器相同的证书。


我们的目标如下:当用户尝试访问 \texttt{example.com} 时,
我们将使该用户进入我们的服务器,该服务器托管一个伪造的 \texttt{example.com}。
如果这是一个社交网络网站,则假网站可以显示类似于目标网站中的登录页面。
如果用户无法分辨出区别,则可以在伪造的网页中键入其帐户凭据,从而使攻击者获得凭据。


\paragraph{第 2 步: 成为中间人}
有几种方法可以使用户的 HTTPS 请求进入我们的 Web 服务器。
一种方法是路由攻击,使用户的 HTTPS 请求被路由到我们的 Web 服务器。
另一种方法是 DNS 攻击,当受害者的计算机尝试找出目标 Web 服务器的 IP 地址时,
它将获取到我们 Web 服务器的 IP 地址。
在此任务中,我们模拟 DNS 攻击。
我们无需发动真正的的 DNS 缓存中毒攻击,只需要修改受害者机器的 \texttt{/etc/hosts} 文件,
通过把 \texttt{www.example.com} 映射到
我们的恶意 Web 服务器上来模拟 DNS 缓存存储攻击的结果。


\begin{lstlisting}
10.9.0.80  www.example.com
\end{lstlisting}


\paragraph{第 3 步: 浏览目标网站}
完成所有设置后,现在访问目标真实网站,并查看您的浏览器会说些什么。
请解释你观察到的现象。



% -------------------------------------------
% SUBSECTION
% -------------------------------------------
\subsection{任务 6: 使用一个被攻陷的 CA 发起中间人攻击}

在本任务中,假设我们在任务 1 中创建的根 CA 被攻击者攻破,并且其私钥被盗。
因此,攻击者可以使用此 CA 的私钥生成任意证书。
在此任务中,我们将看到这种破坏的结果。



请设计一个实验,以表明攻击者可以在任何HTTPS网站上成功发起MITM攻击。
你可以使用在任务5中创建的相同设置,但是这次,你需要证明MITM攻击是成功的。
即当受害人试图访问网站时,浏览器不会有起任何怀疑,而是落入MITM攻击者的虚假网站。




% *******************************************
% SECTION
% *******************************************
\section{提交}

\seedsubmission


%%%%%%%%%%%%%%%%%%%%%%%%%%%%%%%%%%%%%%%%%%%%%%%%%%%%%%
\end{document}
%%%%%%%%%%%%%%%%%%%%%%%%%%%%%%%%%%%%%%%%%%%%%%%%%%%%%%

