%%%%%%%%%%%%%%%%%%%%%%%%%%%%%%%%%%%%%%%%%%%%%%%%%%%%%%%%%%%%%%%%%%%%%%
%%  Copyright by Wenliang Du.                                       %%
%%  This work is licensed under the Creative Commons                %%
%%  Attribution-NonCommercial-ShareAlike 4.0 International License. %%
%%  To view a copy of this license, visit                           %%
%%  http://creativecommons.org/licenses/by-nc-sa/4.0/.              %%
%%%%%%%%%%%%%%%%%%%%%%%%%%%%%%%%%%%%%%%%%%%%%%%%%%%%%%%%%%%%%%%%%%%%%%

\newcommand{\commonfolder}{../../common-files}

\input{\commonfolder/header}
\input{\commonfolder/copyright}


\lhead{\bfseries SEED Labs -- 哈希长度扩展攻击}


\begin{document}


\begin{center}
   {\LARGE 哈希长度扩展攻击}
\end{center}

\seedlabcopyright{2019}


\section{绪论}

当客户端和服务器通过 Internet 通信时,它们会受到中间人(MITM)攻击。
攻击者可以拦截来自客户端的请求,他可能选择修改数据,然后将修改后的请求发送到服务器。
在这种情况下,服务器需要验证收到的请求的完整性。
验证完整性的标准方法是在请求上附加一个称为消息认证码(MAC)的标签。
计算MAC的方法有很多,但其中一些方法并不安全。


MAC 是用密钥和消息计算出来的。
一种简单的计算 MAC 的方法是将密钥与消息连接起来,并计算结果字符串的单向哈希。
这种方法看起来不错,但是会遭受长度扩展攻击。
这种攻击使攻击者可以修改消息,同时仍然可以基于修改后的消息生成有效的MAC,而无需知道密钥。

本实验的目的是帮助学生了解长度扩展攻击的工作原理。
学生将对服务器程序发起攻击;他们将伪造一个有效的命令并让服务器执行该命令。


\paragraph{阅读材料}
对单向哈希函数的详细介绍可以参见:

\begin{itemize}
   \item Chapter 22 of the SEED Book, \seedbook
\end{itemize}


\paragraph{实验环境}
\seedenvironmentB
\nodependency




% *******************************************
% SECTION
% *******************************************
\section{实验环境}

我们为此实验设立了一个 Web 服务器。
客户端可以向该服务器发送一系列的命令。
每个请求必须附加一个使用密钥和命令列表计算出来的的 MAC 。
如果服务器成功验证了 MAC ,它将仅执行请求中的命令。
我们使用主机 VM 作为客户端,并使用一个容器作为 Web 服务器。


\paragraph{容器安装及其命令}
%%%%%%%%%%%%%%%%%%%%%%%%%%%%%%%%%%%%%%%%%%%%
请从实验的网站下载 \texttt{Labsetup.zip} 文件到你的虚拟机中,解压它,
进入 \texttt{Labsetup} 文件夹,然后用 \texttt{docker-compose.yml} 文件安装实验环境。
对这个文件及其包含的所有 \texttt{Dockerfile} 文件中的内容的详细解释都可以
在链接到本实验网站的用户手册 \footnote{
    如果你在部署容器的过程中发现从官方源下载容器镜像非常慢,
    可以参考手册中的说明使用当地的镜像服务器
} 中找到。
如果这是您第一次使用容器设置 SEED 实验环境,那么阅读用户手册非常重要。

在下面,我们列出了一些与 Docker 和 Compose 相关的常用命令。
由于我们将非常频繁地使用这些命令,因此我们在 \texttt{.bashrc} 文件
(在我们提供的 SEEDUbuntu 20.04 虚拟机中)中为它们创建了别名。


\begin{lstlisting}
$ docker-compose build  # Build the container image
$ docker-compose up     # Start the container
$ docker-compose down   # Shut down the container

// Aliases for the Compose commands above
$ dcbuild       # Alias for: docker-compose build
$ dcup          # Alias for: docker-compose up
$ dcdown        # Alias for: docker-compose down
\end{lstlisting}


所有容器都在后台运行。
要在容器上运行命令,我们通常需要获得容器里的 shell 。
首先需要使用 \texttt{docker ps} 命令找出容器的 ID ,
然后使用 \texttt{docker exec} 在该容器上启动 shell 。
我们已经在 \texttt{.bashrc} 文件中为这两个命令创建了别名。

\begin{lstlisting}
$ dockps        // Alias for: docker ps --format "{{.ID}}  {{.Names}}"
$ docksh <id>   // Alias for: docker exec -it <id> /bin/bash

// The following example shows how to get a shell inside hostC
$ dockps
b1004832e275  hostA-10.9.0.5
0af4ea7a3e2e  hostB-10.9.0.6
9652715c8e0a  hostC-10.9.0.7

$ docksh 96
root@9652715c8e0a:/#

// Note: If a docker command requires a container ID, you do not need to
//       type the entire ID string. Typing the first few characters will
//       be sufficient, as long as they are unique among all the containers.
\end{lstlisting}


如果你在设置实验环境时遇到问题,可以尝试从手册的``Miscellaneous Problems''部分中寻找解决方案。

%%%%%%%%%%%%%%%%%%%%%%%%%%%%%%%%%%%%%%%%%%%%



\paragraph{关于 Web 服务器}
我们将服务器程序部署在 \texttt{www.seedlab-hashlen.com} 域名上。
在我们的 VM 中,我们通过将以下条目添加到 \texttt{/etc/hosts} 文件中来实现
将此主机名映射到 Web 服务器容器 \texttt{10.9.0.80}
(如果该条目不在VM中,请添加该条目)。

\begin{lstlisting}
10.9.0.80  www.seedlab-hashlen.com
\end{lstlisting}


服务器代码位于 \path{Labsetup/image_flask/app} 文件夹中。
它有两个目录:
\texttt{www}目录包含服务器代码,
\texttt{LabHome} 目录包含秘密文件和用于计算 MAC 的密钥。




\paragraph{发送请求} 服务器程序接受以下命令:

\begin{itemize}
   \item \texttt{lstcmd} 命令:
         服务器将列出 \texttt{LabHome} 文件夹中的所有文件。
   \item \texttt{download} 命令:
         服务器将从 \texttt{LabHome} 目录返回指定文件的内容。
\end{itemize}

客户端发送到服务器的正常请求如下所示。
客户端需要传递 \texttt{uid} 参数给服务器。
服务器根据 \texttt{uid} 从 \texttt{LabHome/key.txt} 获取 MAC 密钥。
下例中的命令为 \texttt{lstcmd} ,其值设置为 \texttt{1}。
它请求服务器列出所有文件。
最后一个参数是基于密钥(由客户端和服务器共享)和命令参数计算出的 MAC 。
在执行命令之前,服务器将验证 MAC 以确保命令的完整性。

\begin{lstlisting}
http://www.seedlab-hashlen.com/?myname=JohnDoe&uid=1001&lstcmd=1
&mac=dc8788905dbcbceffcdd5578887717c12691b3cf1dac6b2f2bcfabc14a6a7f11
\end{lstlisting}

学生应将 \texttt{myname} 字段中的 \texttt{JohnDoe} 值替换为真实的名字(不允许使用空格)。
此参数是为了确保不同学生的结果不同,因此学生不能互相复制。
服务器不使用此参数,但会检查该参数是否存在。
如果不包含此字段,则请求将被拒绝。
教师可以使用此参数来检查学生是否自己完成了作业。
如果学生在此任务中不使用真实姓名,则没有任何分数。

下面展示了另一个示例。
该请求包括两个命令:列出所有文件并下载文件 \texttt{secret.txt}。
同样需要附加有效的 MAC ,否则服务器将不会执行这些命令。

\begin{lstlisting}
http://www.seedlab-hashlen.com/?myname=JohnDoe&uid=1001&lstcmd=1
&download=secret.txt
&mac=dc8788905dbcbceffcdd5578887717c12691b3cf1dac6b2f2bcfabc14a6a7f11
\end{lstlisting}



% *******************************************
% SECTION
% *******************************************
\section{实验任务}


% -------------------------------------------
% SUBSECTION
% -------------------------------------------
\subsection{任务 1: Send Request to List Files}

在此任务中,我们将向服务器发送一个良性请求,这样我们可以看到服务器如何响应该请求。
我们要发送的请求如下:

\begin{lstlisting}
http://www.seedlab-hashlen.com/?myname=<name>&uid=<need-to-fill>
&lstcmd=1&mac=<need-to-calculate>
\end{lstlisting}

要发送这样的请求,除了使用我们的真实姓名外,我们需要填写两个缺少的参数。
学生需要从 \texttt{LabHome} 目录中的 \texttt{key.txt} 中选择一个 uid 号。
该文件包含冒号分隔的 uid 和密钥的列表。
学生可以使用任何 uid 及其关联的密钥。
例如,学生可以使用 uid \texttt{1001} 及其密钥 \texttt{123456}。

缺少的第二个参数是 MAC ,可以通过将密钥与请求 \texttt{R} 的内容(仅参数部分)连接起来,
并在其之间添加冒号来计算得出。
请参见以下示例:

\begin{lstlisting}
Key:R = 123456:myname=JohnDoe&uid=1001&lstcmd=1
\end{lstlisting}

使用以下命令可以计算 MAC :

\begin{lstlisting}
$ echo -n "123456:myname=JohnDoe&uid=1001&lstcmd=1" | sha256sum
7d5f750f8b3203bd963d75217c980d139df5d0e50d19d6dfdb8a7de1f8520ce3  -
\end{lstlisting}

这样我们就可以构造完整的请求,然后使用浏览器将其发送到服务器程序:

\begin{lstlisting}
http://www.seedlab-hashlen.com/?myname=JohnDoe&uid=1001&lstcmd=1
&mac=7d5f750f8b3203bd963d75217c980d139df5d0e50d19d6dfdb8a7de1f8520ce3
\end{lstlisting}

\paragraph{任务}
请向服务器发送下载命令,并显示可以获得结果。


% -------------------------------------------
% SUBSECTION
% -------------------------------------------
\subsection{任务 2: 构建填充}

要进行哈希长度扩展攻击,我们需要了解如何为单向哈希计算填充。
SHA-256 的块大小为 64 字节,因此在哈希计算过程中,消息 \texttt{M} 将被填充为 64 字节的倍数。
根据 RFC 6234 , SHA256 的填充包括一个字节的 \textbackslash x80 ,后跟多个 0 ,
然后是一个 64 位( 8 字节)的长度字段(长度是 \texttt{M} 中的 \textbf{bits} 数)。

假定原始消息是 \texttt{M = "This is a test message"}。
\texttt{M} 的长度为 \texttt{22} 字节,因此填充为 \texttt{64-22 = 42} 字节,
包括长度字段的 \texttt{8} 字节。
\texttt{M} 的位长度为 \texttt{22 * 8 = 176 = 0xB0}。
SHA256 将在以下填充消息中执行:

\begin{lstlisting}
"This is a test message"
"\x80"
"\x00\x00\x00\x00\x00\x00\x00\x00\x00\x00"
"\x00\x00\x00\x00\x00\x00\x00\x00\x00\x00"
"\x00\x00\x00\x00\x00\x00\x00\x00\x00\x00"
"\x00\x00\x00"
"\x00\x00\x00\x00\x00\x00\x00\xB0"
\end{lstlisting}


需要注意的是,长度字段使用大端序,即如果消息的长度为 \texttt{0x012345} ,
则填充中的长度字段应为:
\begin{lstlisting}
"\x00\x00\x00\x00\x00\x01\x23\x45"
\end{lstlisting}


\paragraph{任务}
学生需要为以下消息构造填充 ( \texttt{<key>} 和 \texttt{<uid>} 的实际值应从
\texttt{LabHome/key.txt} 文件中获取。

\begin{lstlisting}
<key>:myname=<name>&uid=<uid>&lstcmd=1
\end{lstlisting}



% -------------------------------------------
% SUBSECTION
% -------------------------------------------
\subsection{任务 3: 使用密钥计算 MAC}

在本实验中,我们添加一个额外的消息 \texttt{N = "Extra message"}
到填充后的原始消息上 \texttt{M = "This is a test message"},并计算它的哈希值。
该程序如下所示。

\begin{lstlisting}
/* calculate_mac.c */
#include <stdio.h>
#include <openssl/sha.h>

int main(int argc, const char *argv[])
{
   SHA256_CTX c;
   unsigned char buffer[SHA256_DIGEST_LENGTH];
   int i;

   SHA256_Init(&c);
   SHA256_Update(&c,
      "This is a test message"
      "\x80"
      "\x00\x00\x00\x00\x00\x00\x00\x00\x00\x00"
      "\x00\x00\x00\x00\x00\x00\x00\x00\x00\x00"
      "\x00\x00\x00\x00\x00\x00\x00\x00\x00\x00"
      "\x00\x00\x00"
      "\x00\x00\x00\x00\x00\x00\x00\xB0"
      "Extra message",
      64+13);
   SHA256_Final(buffer, &c);

   for(i = 0; i < 32; i++) {
     printf("%02x", buffer[i]);
   }
   printf("\n");
   return 0;
}
\end{lstlisting}

学生可以编译运行上面的程序如下所示:

\begin{lstlisting}
$ gcc calculate_mac.c -o calculate_mac -lcrypto
$ ./calculate_mac
\end{lstlisting}


\paragraph{任务}
学生应该更改上面清单中的代码,并为以下请求计算MAC(假设我们知道MAC密钥):

\begin{lstlisting}
http://www.seedlab-hashlen.com/?myname=<name>&uid=<uid>
&lstcmd=1<padding>&download=secret.txt
&mac=<hash-value>
\end{lstlisting}

就像上一个任务一样, \texttt{<name>} 的值应该是你的真实姓名。
\texttt{<uid>} 和 MAC 密钥的值应从 \texttt{LabHome/key.txt} 文件中获取。
把请求发送到服务器,查看能否成功下载 \texttt{secret.txt} 文件。

应该注意的是,在 URL 中,需要通过将 \texttt{\textbackslash x} 更改为 \texttt{\%}
来对填充中的所有十六进制数字进行编码。
例如,在上面的 URL 中,填充中的 \texttt{\textbackslash x80} 应该替换为 \texttt{\%80}。
在服务器端, URL 中的编码数据将改回十六进制数。


% -------------------------------------------
% SUBSECTION
% -------------------------------------------
\subsection{任务 4: 长度扩展攻击}

在上一个任务中,我们展示了合法用户如何计算 MAC (了解 MAC 密钥)。
在此任务中,我们将以攻击者的身份进行,即,我们不知道MAC密钥。
但是,我们确实知道有效请求 \texttt{R} 的 MAC 。
我们的工作是基于 \texttt{R} 提出新的请求,同时仍然能够计算有效的 MAC 。

给定原始消息 \texttt{M = "This is a test message"} 及其 MAC 值,
我们将说明如何在填充的 \texttt{M} 的末尾添加消息 \texttt{"Extra message"} ,
然后在不知道秘密 MAC 密钥的情况下计算其 MAC 。

\begin{lstlisting}
$ echo -n "This is a test message" | sha256sum
6f3438001129a90c5b1637928bf38bf26e39e57c6e9511005682048bedbef906
\end{lstlisting}

以下程序可用于计算新消息的 MAC :

\begin{lstlisting}[escapechar=|]
/* length_ext.c */
#include <stdio.h>
#include <arpa/inet.h>
#include <openssl/sha.h>

int main(int argc, const char *argv[])
{
  int i;
  unsigned char buffer[SHA256_DIGEST_LENGTH];
  SHA256_CTX c;

  SHA256_Init(&c);
  for(i=0; i<64; i++)
     SHA256_Update(&c, "*", 1);

  // MAC of the original message M (padded)
  c.h[0] = htole32(0x6f343800);
  c.h[1] = htole32(0x1129a90c);
  c.h[2] = htole32(0x5b163792);
  c.h[3] = htole32(0x8bf38bf2);
  c.h[4] = htole32(0x6e39e57c);
  c.h[5] = htole32(0x6e951100);
  c.h[6] = htole32(0x5682048b);
  c.h[7] = htole32(0xedbef906);

  // Append additional message
  SHA256_Update(&c, "Extra message", 13);
  SHA256_Final(buffer, &c);

  for(i = 0; i < 32; i++) {
     printf("%02x", buffer[i]);
  }
  printf("\n");
  return 0;
}
\end{lstlisting}

Students can compile the program as follows:

\begin{lstlisting}
$ gcc length_ext.c -o length_ext -lcrypto
\end{lstlisting}

\paragraph{任务}
学生应首先为以下请求生成有效的 MAC
(其中 \texttt{<uid>} 和 MAC 密钥应从 \texttt{LabHome/key.txt} 文件获得):

\begin{lstlisting}
http://www.seedlab-hashlen.com/?myname=<name>&uid=<uid>
&lstcmd=1&mac=<mac>
\end{lstlisting}

请基于上面计算出的 \texttt{<mac>} 构造一个新的、包含 \texttt{download} 命令的请求。
这一次,你不允许使用密钥。
URL 形如:

\begin{lstlisting}
http://www.seedlab-hashlen.com/?myname=<name>&uid=<uid>
&lstcmd=1<padding>&download=secret.txt&mac=<new-mac>
\end{lstlisting}

把构造出的请求发送给服务器,展示你能成功获得 \texttt{secret.txt} 文件的内容。



% -------------------------------------------
% SUBSECTION
% -------------------------------------------
\subsection{任务 5: 使用 HMAC 缓解攻击}

到目前为止,我们已经看到通过将密钥和消息连接在一起以不安全的方式计算 MAC 时所造成的危害。
在此任务中,我们将改正这一错误。
计算 MAC 的标准方法是使用 HMAC 。
学生应该修改服务器程序的 \texttt{verify\_mac()} 函数,
并使用 Python 的 \texttt{hmac} 模块来计算MAC。
该函数位于 \texttt{lab.py} 中。
给定密钥和消息(均为字符串类型),可以计算 HMAC 如下所示:

\begin{lstlisting}
mac = hmac.new(bytearray(key.encode('utf-8')),
               msg=message.encode('utf-8', 'surrogateescape'),
               digestmod=hashlib.sha256).hexdigest()
\end{lstlisting}

在使用 HMAC 进行 MAC 计算时,学生应重复任务 1 以发送列出文件的请求。
假设选择的密钥是 123456 ,则可以通过以下方式计算 HMAC :

\begin{lstlisting}
$ python3
Python 3.5.2
[GCC 5.4.0 20160609] on linux
Type "help", "copyright", "credits" or "license" for more information
>>> import hmac
>>> import hashlib
>>> key='123456'
>>> message='lstcmd=1'
>>> mac = hmac.new(bytearray(key.encode('utf-8')),
                   msg=message.encode('utf-8', 'surrogateescape'),
                   digestmod=hashlib.sha256).hexdigest()
\end{lstlisting}

学生应说明为什么当客户端和服务器使用 HMAC 时,
使用长度扩展和附加命令的恶意请求将使 MAC 验证失败。

% *******************************************
% SECTION
% *******************************************
\section{Submission}

%%%%%%%%%%%%%%%%%%%%%%%%%%%%%%%%%%%%%%%%
\input{\commonfolder/submission}
%%%%%%%%%%%%%%%%%%%%%%%%%%%%%%%%%%%%%%%%


\end{document}
